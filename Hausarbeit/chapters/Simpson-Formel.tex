\chapter{Simpson-Formel, Varianten sowie Unterschiede und wo welche Simpson Formel sinnvoll ist}

\section{Varianten}

Bei der Simpson Regel gibt es vier verschiedene Varianten, welche nochmal in zwei Kategorien aufgeilt werden können. Man kann es in die \(\frac{1}{3}\)-Formel und die \(\frac{3}{8}\)-Formel unterteilen. Diese kann man dann wiederrum in die Einfache Simpson-\(\frac{1}{3}\)-Formel und die Zusammengesetzte Simpson-\(\frac{1}{3}\)-Formel, sowie die Einfache-\(\frac{3}{8}\)-Formel und die Zusammengesetzte Simpson-\(\frac{3}{8}\)-Formel unterteilen \textsc{\cite[S. 342]{SimpsonVarianten}}  \textsc{\cite[S. 327]{BasicNumMath}}.


\section{Unterschiede}\label{Unterschiede}

Bei den Unterschieden kann man ebenfalls wieder in die zwei Kategorien \(\frac{1}{3}\)-Formel und \(\frac{3}{8}\)-Formel unterteilen. Die Einfache Simpson-\(\frac{1}{3}\)-Formel kann man für ein einzelnes Intervall benutzen, welches in zwei gleich große Teile geteilt wurde. Die Zusammengesetzte Simpson-\(\frac{1}{3}\)-Formel kann man dahingegen für ein Intervall nutzen, welches in mehrere gleich große Teile geteilt wurde, wobei die Anzahl durch zwei teilbar sein muss. Bei den \(\frac{3}{8}\)-Formeln ist das so ähnlich, bis auf enen kleinen Unterschied. Die Einfache-\(\frac{3}{8}\)-Formel nutzt man für ein einzelnes Intervall, welches in drei gleich große Teile geteilt wurde und die Zusammengesetzte Simpson-\(\frac{3}{8}\)-Formel nutzt man für ein Intervall, das in mehrere gleich große Teile geteilt wurde, allerdings muss die Anzahl der Teile hierbei durch drei teilbar sein \textsc{\cite[S. 310]{NumMathe}} \textsc{\cite[S. 180]{NumMethodsScienceEng}}.


\section{Unter welchen Bedingungen bzw. Anforderungen ist welche Simpson Formel sinnvoll}

Im allgemeinen dienen die Simpson-Formeln dazu, eine höhere Genauigkeit zu erreichen als beispielsweise mit der Rechtecks- oder Trapezberechnung. \textsc{\cite[S. 310]{NumMathe}} Sofern die Abstände der Punkte gleichmäßig sind, kann man Formeln, wie die Simpson-Formeln, höherer Ordnung anwenden. Sowohl die \(\frac{1}{3}\)-Formel und die \(\frac{3}{8}\)-Formel haben die gleiche Genauigkeit. Man kann die Simpson-Formeln allerdings nicht nutzen, wenn die Punkte nicht gleichmäßig vertielt sind \textsc{\cite[S. 186]{NumMethodsScienceEng}}. Wie bei den Unterschieden \ref{Unterschiede} erwähnt, kann man die \(\frac{1}{3}\)-Formel dafür benutzen, um eine Fläche in einem Intervall zu berechnen, welche aus einer geraden Anzahl von Teilintervallen besteht. Hierbei muss man allerdings darauf achten, dass die Einfache Simpson-\(\frac{1}{3}\)-Formel nur zwei Teilintervalle haben darf, dafür aber die Zusammengesetzte Simpson-\(\frac{1}{3}\)-Formel aus mehreren Teilintervallen bestehen darf, diese Anzahl aber durch zwei teilbar sein muss. Bei der \(\frac{3}{8}\)-Formel ist das ähnlich, aber mit drei. Die Einfache Simpson-\(\frac{3}{8}\)-Formel muss aus drei Teilintervallen bestehen und die Zusammengesetzte Simpson-\(\frac{3}{8}\)-Formel muss aus einer Anzahl an Teilintervallen bestehen, welche durch drei Teilbar ist \textsc{\cite[S. 180]{NumMethodsScienceEng}} \textsc{\cite[S. 186]{NumMethodsScienceEng}}. 








