\chapter{Wo spielen Differentiation und Integration in der Praxis eine Rolle und was sind die Schwierigkeiten dabei?}

\section{Praxisanwendungen Integralrechnung}

In vielen Bereichen spielt die praktischen Anwendung der Integralrechnung oft eine wichtige Rolle. In den Ingenieurswissenschaften oder in der Physik wird die Integralrechnung unter anderem zum Berechnen verschiedenster Flächen unter willkürlichen Kurven und zum Berechnen von Volumina genutzt \textsc{\cite{ARD-Alpha-Integral}} \textsc{\cite{Anwendung-Integrale}}. Des Weiteren kann man mit der Integralrechnung in der Mechanik verschiedene physikalische Größen, wie zum Beispiel die Schnittgröße berechnen \textsc{\cite{TM-Schnittgroessen}}. Die Integralrechnung spielt nicht nur in der Physik und den Ingenieurswissenschaften eine wichtige Rolle, sondern auch in der Statistik und Wahrscheinlichkeitsrechnung. Hierbei werden Integrale genutzt um den Erwartungswert einer Zufallszahl zu berechnen \textsc{\cite{MatheIngVertiefung_S.520f}}. Allerdings spielt nicht nur in den sehr mathematischen Bereichen, sondern auch beispielsweise in Umweltwissenschaften und in der Geographie. In diesen Bereichen wird die Integralrechnung dafür genutzt, um zum Beispiel die Durchflussmenge anhand der Geschwindigkeit über einen gewissen Querschnitt zu berechnen. \textsc{\cite[S. 22]{Hydrologie_S.22}}

\section{Praxisanwendung Differentialrechnung}

Die Differentialrechnung wird ebenfalls, wie die Integralrechnung, in vielen Bereichen von Ingeneieurswissenschaften genutzt um beispielsweise die Änderungsrate an einem exakten Punkt einer Kurve festzustellen, dies kann somit unter anderem die Geschwindigkeit oder das Wachstum zu einer bestimmten Zeit sein. \textsc{\cite{ElektroAbleitung}}

\section{Schwierigkeiten bei der Integral- und Differentialrechnung}

Die Schwierigkeiten oder Problematik bei der Integralrechnung liegt darin, dass bei numerischen Integrationsmethoden, wie der Sehnen-Trapez oder bei der Simpson-Methode, eine niedrige Genauigkeit herrscht \textsc{\cite[S. 205]{WingMatheProblem_S218}}. Des Weiteren haben viele Funktionen in der Praxis keine einfachen, beziehungsweise keine bekannten Stammfunktionen. Dies erschwert die Integration dieser Funktion erheblich.
Die Schwierigkeiten bei der Differentialrechnung bestehen darin, dass zum einen, wie bei der Integralrechnung auch, die Funktionen nicht immer einfach sind und zum Anderen, dass Studenten oder Schüler oft die Steigung nicht direkt mit der Ändereungsrate in Verbindung setzen \textsc{\cite[S. 78]{AbleitungMatheWingWis_S78}}. 
